\usepgflibrary{shapes.arrows}

% Make font for captions smaller (saves space and visually separates them from the main text)
\usepackage[font={footnotesize,sf}]{caption}

% Print menu and key command descriptions
\usepackage{menukeys}

% make underscore work without escaping in text mode
\usepackage{underscore}

% Common style that will be used for titles
\newcommand{\titlestyle}{\color{DodgerBlue3}\sffamily}

% Perfect 1-inch margins
\usepackage[margin=1in]{geometry}
\usepackage[medium,compact]{titlesec}
\usepackage{longtable}
\titleformat*{\section}{\titlestyle \Large}
\titleformat*{\subsection}{\titlestyle \large \itshape}
\titleformat*{\subsubsection}{\titlestyle \large \itshape}

% Print menu and key command descriptions
%\usepackage{menukeys}

%Lists (requires enumitem package):
%Font for description
\setlist[description]{font=\color{DodgerBlue3}}
% Compact lists
\setlist[1]{\labelindent=\parindent}
\setlist[1]{topsep=3pt, itemsep=0pt}

% Format main title
\let\prevtitle\title
\renewcommand{\title}[1]{\prevtitle{\titlestyle #1}}
\let\prevauthor\author
\renewcommand{\author}[1]{\prevauthor{\titlestyle #1}}
\let\prevdate\date
\renewcommand{\date}[1]{\prevdate{\titlestyle #1}}

% Displayed quotations (used to define an environment for function specification)
\usepackage{csquotes}

% Include code
% Include code and org files
\usepackage{listings}
% Smaller fixed-width font, automatic breaking of long lines, ignore indentation in list environments
\lstset{basicstyle=\footnotesize\ttfamily,breaklines=true,resetmargins=true}

%%% Local Variables:
%%% mode: latex
%%% End:


% formatting for the description of a function's input/output arguments
\mdfdefinestyle{functionspecification}{%
  linecolor=DodgerBlue3,
  rightline=false, topline=false, bottomline=false,
  outerlinewidth=1.35,
  roundcorner=10pt,
  innerleftmargin=1.2em,
  innertopmargin=0.3em,
  innerbottommargin=0.3em,
  leftmargin=1pt,
  skipabove=0.5em,
  skipbelow=0.5em,
  backgroundcolor=white}

\newcommand{\dstar}{$\bigstar$}

% Clever cross-references
\usepackage[capitalize,nameinlink]{cleveref}

%Call sections as Problems, color paragraph titles
\titleformat{\section}{\titlestyle\large\bfseries}{Problem \thesection:}{0.5ex}{}
%\titleformat{\subsection}{\normalfont\bfseries}{Problem \thesubsection:}{0.5ex}{}
\titleformat{\paragraph}[runin]{\titlestyle\normalsize\bfseries}{}{2pt}{}

%TikZ styles for drawing graphs of ROS nodes
\tikzset{
ros node/.style={draw,rounded corners=4mm,inner sep=3mm,font=\small,line width=2pt},
provided/.style={draw=Orange1},
provided node/.style={ros node,provided},
code/.style={draw=SlateBlue2},
code node/.style={ros node,code},
previous/.style={draw=SlateBlue2!50},
previous node/.style={ros node,previous}
}

%%% Local Variables:
%%% mode: latex
%%% End:

\usepackage{probsoln}
\tikzset{ax/.style={very thick,-latex},pnt/.style={minimum size=5pt,fill=blue,circle,inner sep=0pt}}

%ref: https://tex.stackexchange.com/questions/201197/latex-using-the-matching-environment-in-the-exam-documentclass
\usepackage{multicol}

\newcommand{\vspaceboxdelta}{3mm}
\newcommand{\vspacebox}[1]{\fullwidth{\tikz{\useasboundingbox (0,-\vspaceboxdelta) rectangle (\textwidth,#1\baselineskip); \draw[gray!20!white,line width=3pt] (-\vspaceboxdelta,-\vspaceboxdelta) rectangle (\textwidth+2*\vspaceboxdelta,#1\baselineskip+\vspaceboxdelta);}}\par}
\newcommand{\vspaceoneline}{\vspacebox{2}}
\newcommand{\vspacetwolines}{\vspacebox{3}}
\newcommand{\vspacethreelines}{\vspacebox{4}}
\newcommand{\vspacefourlines}{\vspacebox{5}}
\newcommand{\vspacefivelines}{\vspacebox{6}}
\newcommand{\vspacesixlines}{\vspacebox{7}}
\newcommand{\vspacesevenlines}{\vspacebox{8}}

%% Create a Matching question format
\newcommand*\Matching[1]{
\ifprintanswers
    \textbf{#1}
\else
    \rule{2.7in}{0.5pt}
\fi
}
\newlength\matchlena
\newlength\matchlenb
\settowidth\matchlena{\rule{2.7in}{0pt}}
\newcommand\MatchQuestion[2]{%
    \setlength\matchlenb{\linewidth}
    \addtolength\matchlenb{-\matchlena}
    \parbox[t]{\matchlena}{\Matching{#1}}\enspace\parbox[t]{\matchlenb}{#2}}

%%% Local Variables:
%%% mode: latex
%%% End:
